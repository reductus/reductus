\documentclass[12pt,letterpaper]{article}
\author{Brian B. Maranville}
\title{Polarization correction for neutron reflectometry with transmission spin filters}
\usepackage[utf8]{inputenc}
\usepackage[margin=1.0in]{geometry}
\usepackage{amsmath}
\usepackage{hyperref}
\usepackage{amsfonts}
\usepackage{amssymb}
\usepackage{mathtools}
\begin{document}
\maketitle

\section{Adaptation from source}
Adaptation of polarization efficiency corrections from 
``Neutron scattering studies of magnetic thin films and multilayers''

C.F. Majkrzak (1996). Physica B 221, 342-356

\url{http://dx.doi.org/10.1016/0921-4526(95)00948-5}
\subsection{flipper efficiency}
For any neutron-polarizing optical system, a spin-filter is chosen which selects a particular spin
state which we will call the dominant state.

The transmitted intensity of the dominant spin state through the polarizing optics (at the sample)
is proportional to $(1 + F)$ when the front flipper is off and $(1 - F(1 - 2f))$ when the flipper is on; 
the transmitted intensity of the other (leakage) spin state is $(1 - F)$ for flipper off and $(1 + F(1 - 2f))$
for flipper on, where $F, f$ are the efficiencies of the polarizer device and the flipper, respectively.
$F$ and $f$ are defined such that $0 \leq F \leq 1$ and $0 \leq f \leq 1$, so that in the above
equations, for a perfect polarized and flipper the transmission would be 1 for the dominant spin state
with the flipper off, with 0 leakage, and vice-versa with the flipper off.  The same set of equations holds
for the transmission through the spin-analyzer optics after the beam interacts with the sample: the analyzer
efficiency and rear flipper efficiency will be written as $R, r$.

In the reference above, the math is defined for a supermirror polarizer and analyzer in the reflection geometry.
This corresponds to a dominant transmitted neutron spin state of $(+)$ when a flipper is 
off and $(-)$ when a flipper is on.

\subsection{spin-filter geometry: transmit or reflect}
To apply the equations in Table 2 of that reference to a system in which the supermirrors are
operated in a transmission geometry, we have to reverse the sign of all the spin-dependent cross-sections described,
(all the subscripts on the $\sigma$) while keeping the relationship of transmission efficiencies to the flipper state
(off vs. on) the same, e.g.
\begin{equation}
\begin{array}{rll}
	I^{\textrm{off off}}_\textrm{reflect} / \beta =& {} & \sigma_{++} (1 + F)(1 + R) \\
	 {} & + & \sigma_{-+} (1 - F)(1 + R) \\
	 {} & + & \sigma_{--} (1 - F)(1 - R) \\
	 {} & + & \sigma_{+-} (1 + F)(1 - R) \\
	 {} & {} & {} \\
	 I^{\textrm{off on}}_\textrm{reflect} / \beta =& {} & \sigma_{++} (1 + F)[1 + R(1-2r)] \\
	 {} & + & \sigma_{-+} (1 - F)[1 + R(1-2r)] \\
	 {} & + & \sigma_{--} (1 - F)[1 - R(1-2r)] \\
	 {} & + & \sigma_{+-} (1 + F)[1 - R(1-2r)] \\
\end{array}
\end{equation}
for a reflection polarizer/analyzer in the reference becomes
\begin{equation}
\begin{array}{rll}
	I^{\textrm{off off}}_\textrm{transmit} / \beta =& {} & \sigma_{--} (1 + F)(1 + R) \\
	 {} & + & \sigma_{+-} (1 - F)(1 + R) \\
	 {} & + & \sigma_{++} (1 - F)(1 - R) \\
	 {} & + & \sigma_{-+} (1 + F)(1 - R) \\
	 {} & {} & {} \\
	 I^{\textrm{off on}}_\textrm{transmit} / \beta =& {} & \sigma_{--} (1 + F)[1 + R(1-2r)] \\
	 {} & + & \sigma_{+-} (1 - F)[1 + R(1-2r)] \\
	 {} & + & \sigma_{++} (1 - F)[1 - R(1-2r)] \\
	 {} & + & \sigma_{-+} (1 + F)[1 - R(1-2r)] \\
\end{array}
\end{equation}
and similarly for the other two flipper states...
\begin{equation}
\begin{array}{rll}
	I^{\textrm{on off}}_\textrm{transmit} / \beta =& {} & \sigma_{--} [1 + F(1-2f)](1 + R) \\
	 {} & + & \sigma_{+-} [1 - F(1-2f)](1 + R) \\
	 {} & + & \sigma_{++} [1 - F(1-2f)](1 - R) \\
	 {} & + & \sigma_{-+} [1 + F(1-2f)](1 - R) \\
	 {} & {} & {} \\
	 I^{\textrm{on on}}_\textrm{transmit} / \beta =& {} & \sigma_{--} [1 + F(1-2f)][1 + R(1-2r)] \\
	 {} & + & \sigma_{+-} [1 - F(1-2f)][1 + R(1-2r)] \\
	 {} & + & \sigma_{++} [1 - F(1-2f)][1 - R(1-2r)] \\
	 {} & + & \sigma_{-+} [1 + F(1-2f)][1 - R(1-2r)] \\
\end{array}
\end{equation}
For a transmitting polarizer/analyzer, we will identify the measured cross-sections as 
\begin{equation}
	\mathbf{I} = 
	\begin{pmatrix}
		I^{++} \\
		I^{-+} \\
		I^{+-} \\
		I^{--} 
	\end{pmatrix}
	= 
	\begin{pmatrix}
		I^\textrm{on on} \\
		I^\textrm{off on} \\
		I^\textrm{on off} \\
		I^\textrm{off off} 
	\end{pmatrix}_\textrm{transmit}
\end{equation}

\subsection{matrix for transmission spin-filters}
Then re-arranging the cross-sections from the reference into 
$\{(++), (+-), (-+), (--)\}$ (our preferred order) 
we can write the equations as a matrix:
\begin{equation}
\begin{array}{c}
	 A \cdot \boldsymbol{\sigma} = \mathbf{I} \\[1em]

  A = 
  \begin{pmatrix*}[l]
	(1-Fx)(1-Ry) &\!\!\! (1-Fx)(1+Ry) &\!\!\! (1+Fx)(1-Ry) &\!\!\! (1+Fx)(1+Ry) \\
	(1-Fx)(1-R) &\!\!\! (1-Fx)(1+R) &\!\!\! (1+Fx)(1-R) &\!\!\! (1+Fx)(1+R) \\
	(1-F)(1-Ry) &\!\!\! (1-F)(1+Ry) &\!\!\! (1+F)(1-Ry) &\!\!\! (1+F)(1+Ry) \\
	(1-F)(1-R) &\!\!\! (1-F)(1+R) &\!\!\! (1+F)(1-R) &\!\!\! (1+F)(1+R) 
  \end{pmatrix*}
\end{array}
\end{equation}
where $x = (1 - 2f), y = (1-2r)$.  From here, we can invert $A$ to get back the intrinsic
scattering cross-sections $\boldsymbol{\sigma}$ by
\begin{equation}
	\boldsymbol{\sigma} = A^{-1} \cdot \mathbf{I}
\end{equation}

\subsection{reduced matrices}
At times, when measuring some assumptions are made about the samples in question which can
simplify the equations and reduce the measurement time
\subsubsection{spin-flip cross-sections equal, nonzero}

In this case, $\sigma_\textrm{SF} \equiv \sigma_{+-} = \sigma_{-+}$ and there are only three
unknowns to solve for.  We can then measure just one spin-flip (SF) cross-section and solve for
the reflectivity $\boldsymbol{\sigma}$
\begin{equation}
\begin{array}{c}
	A^{+-} \cdot 
	\begin{pmatrix}
	 	\sigma_{++} \\
	 	\sigma_\textrm{SF} \\
	 	\sigma_{--} 
	 \end{pmatrix}	 
	 = 
	 \begin{pmatrix}
		I^{++} \\
		I^{+-} \\
		I^{--} 
	\end{pmatrix}  \\[2em]

  A^{+-} = 
  \begin{pmatrix*}[l]
	(1-Fx)(1-Ry) &\!\!\! [(1-Fx)(1+Ry) + (1+Fx)(1-Ry)] &\!\!\! (1+Fx)(1+Ry) \\
	(1-Fx)(1-R) &\!\!\! [(1-Fx)(1+R) + (1+Fx)(1-R)] &\!\!\! (1+Fx)(1+R) \\
	(1-F)(1-R) &\!\!\! [(1-F)(1+R) + (1+F)(1-R)] &\!\!\! (1+F)(1+R) 
  \end{pmatrix*}
\end{array}
\end{equation}
or similarly
\begin{equation}
\begin{array}{c}
	A^{-+} \cdot 
	\begin{pmatrix}
	 	\sigma_{++} \\
	 	\sigma_\textrm{SF} \\
	 	\sigma_{--} 
	 \end{pmatrix}	 
	 = 
	 \begin{pmatrix}
		I^{++} \\
		I^{-+} \\
		I^{--} 
	\end{pmatrix}  \\[2em]

  A^{-+} = 
  \begin{pmatrix*}[l]
	(1-Fx)(1-Ry) &\!\!\! [(1-Fx)(1+Ry) + (1+Fx)(1-Ry)] &\!\!\! (1+Fx)(1+Ry) \\
	(1-F)(1-Ry) &\!\!\! [(1-F)(1+Ry) + (1+F)(1-Ry)] &\!\!\! (1+F)(1+Ry) \\
	(1-F)(1-R) &\!\!\! [(1-F)(1+R) + (1+F)(1-R)] &\!\!\! (1+F)(1+R) 
  \end{pmatrix*}
\end{array}
\end{equation}

\subsubsection{spin-flip cross-sections zero}

If the spin-flip cross-sections are expected to be vanishingly small 
(which occurs when the in-plane magnetization is strictly parallel or antiparallel to
the field quantization axis) then a further simplification is possible:
\begin{equation}
\begin{array}{c}
	A^\textrm{NSF} \cdot 
	\begin{pmatrix}
	 	\sigma_{++} \\
	 	\sigma_{--} 
	 \end{pmatrix}	 
	 = 
	 \begin{pmatrix}
		I^{++} \\
		I^{--} 
	\end{pmatrix}  \\[2em]

  A^\textrm{NSF} = 
  \begin{pmatrix*}[l]
	(1-Fx)(1-Ry) &\!\!\!  (1+Fx)(1+Ry) \\
	(1-F)(1-R) &\!\!\! (1+F)(1+R) 
  \end{pmatrix*}
\end{array}
\end{equation}

\subsection{Calibration}
In order to do the polarization correction as described, one has to first determine the parameters
$F,R,f,r$.  These are done exactly as described in the reference; the only thing to note is the 
identification of the measured cross-sections for transmission spin-filters:  
\begin{equation}
I^{--} \equiv I^\textrm{off off}_\textrm{NS,transmit} = \alpha[FR + 1]
\end{equation}
\begin{equation}
I^{+-} \equiv I^\textrm{on off}_\textrm{NS,transmit} = \alpha[FR(1-2f) + 1]
\end{equation}
\begin{equation}
I^{-+} \equiv I^\textrm{off on}_\textrm{NS,transmit} = \alpha[FR(1-2r) + 1]
\end{equation}

\end{document}